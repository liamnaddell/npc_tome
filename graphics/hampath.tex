WTS: 3SAT $\leq_p$ HAMPATH


To do this, we need to take 3-cnf formulas $\phi$ and turn them into paths which contain a hamiltonian path if and only if $\phi$ is satisfiable.


The way we do this is with two primary structures: The variable widget and the clause widget. The structure of the variable widget is such that a truth assignment is given to going left or right. 



\begin{tikzpicture}

\tikzset{vertex/.style = {shape=circle,draw,minimum size=1.5em}}
\tikzset{edge/.style = {->,> = latex'}}
% vertices
\node[vertex] (a) at  (0,0) {};
\node[vertex] (b) at  (4,3) {$s=x_1$};
\node[vertex] (c) at  (8,0) {};
\node[vertex] (d) at  (4,-3) {};
\node[vertex] (a1) at (1.5,0) {};
\node[vertex] (a2) at (3,0) {};
%edges
\draw[edge] (b) to (a);
\draw[edge] (b) to (c);
\draw[edge] (a) to (d);
\draw[edge] (c) to (d);

\draw[edge] (a)  to[bend left] (a1);
\draw[edge] (a1) to[bend left] (a);

\draw[edge] (a1) to[bend left] (a2);
\draw[edge] (a2) to[bend left] (a1);

\path (a2) to node {\dots} (c);
\node [shape=circle,minimum size=1.5em] (a3) at (4.5,0) {};
\draw[edge] (a2) to[bend left] (a3);
\draw[edge] (a3) to[bend left] (a2);

\node [shape=circle,minimum size=1.5em] (c1) at (6.5,0) {};
\draw[edge] (c) to[bend left] (c1);
\draw[edge] (c1) to[bend left] (c);
\end{tikzpicture}

The number of these ``vertices'' in the middle is equal to 3 times the number of clauses. 3 times allows for 1 buffer node between pairs 

If a variable $x_i$ is in clause $c_j$, then add an edge from the jth pair in the ith diamond to the jth clause node. But, if a variable $\bar x_i$ appears in clause $c_j$, then add two edges going in the reverse direction.

Finally, let G = the graph we have just constructed, and let s be the first variable widget, and t be the last variable widget. This is how we construct G,s, and t. 

How this works: One can only assign 1 value (left/right or true/false) to each variable widget, and each clause widget must be gone through at least once by a single variable. This simulates a truth assignment, and that that truth assignment assigns true to each clause iff there is a satisfiable truth assignment for $\phi$. 

\subsubsection{If $\left<\phi\right>\in\text{3SAT}$}

Then $\phi$ has a satisfiable truth assignment $\tau$. Following the variable widget convention, we will go left/right according to what the truth assignment tells us to do. When we encounter the option to visit a clause, we will take it unless the clause has already been satisfied (i.e. visited). Since this formula is satisfiable, all clause nodes will be hit, as well as the top and bottom of each variable widget, and all nodes inbetween. This implies that $\left<G,s,t\right>\in\text{HAMPATH}$.

\subsubsection{If $\left<G,s,t\right>\in\text{HAMPATH}$}

Then G has a hamiltonian path from s to t, where s is the first variable, and t is the last variable in a truth assignment $\tau$. This hamilton path must cross through each clause widget once, and must cross through each variable widget once. This implies that there must be a truth assignment such that every clause is satisfied, which implies that $\tau$ is a satisfiable truth assignment for $\phi$. Therefore, $\left<\phi\right>\in\text{3SAT}$
